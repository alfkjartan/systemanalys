
% Default to the notebook output style

    


% Inherit from the specified cell style.




    
\documentclass{article}

    
    
    \usepackage{graphicx} % Used to insert images
    \usepackage{adjustbox} % Used to constrain images to a maximum size 
    \usepackage{color} % Allow colors to be defined
    \usepackage{enumerate} % Needed for markdown enumerations to work
    \usepackage{geometry} % Used to adjust the document margins
    \usepackage{amsmath} % Equations
    \usepackage{amssymb} % Equations
    \usepackage{eurosym} % defines \euro
    \usepackage[mathletters]{ucs} % Extended unicode (utf-8) support
    \usepackage[utf8x]{inputenc} % Allow utf-8 characters in the tex document
    \usepackage{fancyvrb} % verbatim replacement that allows latex
    \usepackage{grffile} % extends the file name processing of package graphics 
                         % to support a larger range 
    % The hyperref package gives us a pdf with properly built
    % internal navigation ('pdf bookmarks' for the table of contents,
    % internal cross-reference links, web links for URLs, etc.)
    \usepackage{hyperref}
    \usepackage{longtable} % longtable support required by pandoc >1.10
    \usepackage{booktabs}  % table support for pandoc > 1.12.2
    \usepackage[normalem]{ulem} % ulem is needed to support strikethroughs (\sout)
    

    
    
    \definecolor{orange}{cmyk}{0,0.4,0.8,0.2}
    \definecolor{darkorange}{rgb}{.71,0.21,0.01}
    \definecolor{darkgreen}{rgb}{.12,.54,.11}
    \definecolor{myteal}{rgb}{.26, .44, .56}
    \definecolor{gray}{gray}{0.45}
    \definecolor{lightgray}{gray}{.95}
    \definecolor{mediumgray}{gray}{.8}
    \definecolor{inputbackground}{rgb}{.95, .95, .85}
    \definecolor{outputbackground}{rgb}{.95, .95, .95}
    \definecolor{traceback}{rgb}{1, .95, .95}
    % ansi colors
    \definecolor{red}{rgb}{.6,0,0}
    \definecolor{green}{rgb}{0,.65,0}
    \definecolor{brown}{rgb}{0.6,0.6,0}
    \definecolor{blue}{rgb}{0,.145,.698}
    \definecolor{purple}{rgb}{.698,.145,.698}
    \definecolor{cyan}{rgb}{0,.698,.698}
    \definecolor{lightgray}{gray}{0.5}
    
    % bright ansi colors
    \definecolor{darkgray}{gray}{0.25}
    \definecolor{lightred}{rgb}{1.0,0.39,0.28}
    \definecolor{lightgreen}{rgb}{0.48,0.99,0.0}
    \definecolor{lightblue}{rgb}{0.53,0.81,0.92}
    \definecolor{lightpurple}{rgb}{0.87,0.63,0.87}
    \definecolor{lightcyan}{rgb}{0.5,1.0,0.83}
    
    % commands and environments needed by pandoc snippets
    % extracted from the output of `pandoc -s`
    \providecommand{\tightlist}{%
      \setlength{\itemsep}{0pt}\setlength{\parskip}{0pt}}
    \DefineVerbatimEnvironment{Highlighting}{Verbatim}{commandchars=\\\{\}}
    % Add ',fontsize=\small' for more characters per line
    \newenvironment{Shaded}{}{}
    \newcommand{\KeywordTok}[1]{\textcolor[rgb]{0.00,0.44,0.13}{\textbf{{#1}}}}
    \newcommand{\DataTypeTok}[1]{\textcolor[rgb]{0.56,0.13,0.00}{{#1}}}
    \newcommand{\DecValTok}[1]{\textcolor[rgb]{0.25,0.63,0.44}{{#1}}}
    \newcommand{\BaseNTok}[1]{\textcolor[rgb]{0.25,0.63,0.44}{{#1}}}
    \newcommand{\FloatTok}[1]{\textcolor[rgb]{0.25,0.63,0.44}{{#1}}}
    \newcommand{\CharTok}[1]{\textcolor[rgb]{0.25,0.44,0.63}{{#1}}}
    \newcommand{\StringTok}[1]{\textcolor[rgb]{0.25,0.44,0.63}{{#1}}}
    \newcommand{\CommentTok}[1]{\textcolor[rgb]{0.38,0.63,0.69}{\textit{{#1}}}}
    \newcommand{\OtherTok}[1]{\textcolor[rgb]{0.00,0.44,0.13}{{#1}}}
    \newcommand{\AlertTok}[1]{\textcolor[rgb]{1.00,0.00,0.00}{\textbf{{#1}}}}
    \newcommand{\FunctionTok}[1]{\textcolor[rgb]{0.02,0.16,0.49}{{#1}}}
    \newcommand{\RegionMarkerTok}[1]{{#1}}
    \newcommand{\ErrorTok}[1]{\textcolor[rgb]{1.00,0.00,0.00}{\textbf{{#1}}}}
    \newcommand{\NormalTok}[1]{{#1}}
    
    % Additional commands for more recent versions of Pandoc
    \newcommand{\ConstantTok}[1]{\textcolor[rgb]{0.53,0.00,0.00}{{#1}}}
    \newcommand{\SpecialCharTok}[1]{\textcolor[rgb]{0.25,0.44,0.63}{{#1}}}
    \newcommand{\VerbatimStringTok}[1]{\textcolor[rgb]{0.25,0.44,0.63}{{#1}}}
    \newcommand{\SpecialStringTok}[1]{\textcolor[rgb]{0.73,0.40,0.53}{{#1}}}
    \newcommand{\ImportTok}[1]{{#1}}
    \newcommand{\DocumentationTok}[1]{\textcolor[rgb]{0.73,0.13,0.13}{\textit{{#1}}}}
    \newcommand{\AnnotationTok}[1]{\textcolor[rgb]{0.38,0.63,0.69}{\textbf{\textit{{#1}}}}}
    \newcommand{\CommentVarTok}[1]{\textcolor[rgb]{0.38,0.63,0.69}{\textbf{\textit{{#1}}}}}
    \newcommand{\VariableTok}[1]{\textcolor[rgb]{0.10,0.09,0.49}{{#1}}}
    \newcommand{\ControlFlowTok}[1]{\textcolor[rgb]{0.00,0.44,0.13}{\textbf{{#1}}}}
    \newcommand{\OperatorTok}[1]{\textcolor[rgb]{0.40,0.40,0.40}{{#1}}}
    \newcommand{\BuiltInTok}[1]{{#1}}
    \newcommand{\ExtensionTok}[1]{{#1}}
    \newcommand{\PreprocessorTok}[1]{\textcolor[rgb]{0.74,0.48,0.00}{{#1}}}
    \newcommand{\AttributeTok}[1]{\textcolor[rgb]{0.49,0.56,0.16}{{#1}}}
    \newcommand{\InformationTok}[1]{\textcolor[rgb]{0.38,0.63,0.69}{\textbf{\textit{{#1}}}}}
    \newcommand{\WarningTok}[1]{\textcolor[rgb]{0.38,0.63,0.69}{\textbf{\textit{{#1}}}}}
    
    
    % Define a nice break command that doesn't care if a line doesn't already
    % exist.
    \def\br{\hspace*{\fill} \\* }
    % Math Jax compatability definitions
    \def\gt{>}
    \def\lt{<}
    % Document parameters

    \title{Introduction to Python}
    \author{Kjartan Halvorsen}
    
    

    % Pygments definitions
    
\makeatletter
\def\PY@reset{\let\PY@it=\relax \let\PY@bf=\relax%
    \let\PY@ul=\relax \let\PY@tc=\relax%
    \let\PY@bc=\relax \let\PY@ff=\relax}
\def\PY@tok#1{\csname PY@tok@#1\endcsname}
\def\PY@toks#1+{\ifx\relax#1\empty\else%
    \PY@tok{#1}\expandafter\PY@toks\fi}
\def\PY@do#1{\PY@bc{\PY@tc{\PY@ul{%
    \PY@it{\PY@bf{\PY@ff{#1}}}}}}}
\def\PY#1#2{\PY@reset\PY@toks#1+\relax+\PY@do{#2}}

\expandafter\def\csname PY@tok@gd\endcsname{\def\PY@tc##1{\textcolor[rgb]{0.63,0.00,0.00}{##1}}}
\expandafter\def\csname PY@tok@gu\endcsname{\let\PY@bf=\textbf\def\PY@tc##1{\textcolor[rgb]{0.50,0.00,0.50}{##1}}}
\expandafter\def\csname PY@tok@gt\endcsname{\def\PY@tc##1{\textcolor[rgb]{0.00,0.27,0.87}{##1}}}
\expandafter\def\csname PY@tok@gs\endcsname{\let\PY@bf=\textbf}
\expandafter\def\csname PY@tok@gr\endcsname{\def\PY@tc##1{\textcolor[rgb]{1.00,0.00,0.00}{##1}}}
\expandafter\def\csname PY@tok@cm\endcsname{\let\PY@it=\textit\def\PY@tc##1{\textcolor[rgb]{0.25,0.50,0.50}{##1}}}
\expandafter\def\csname PY@tok@vg\endcsname{\def\PY@tc##1{\textcolor[rgb]{0.10,0.09,0.49}{##1}}}
\expandafter\def\csname PY@tok@vi\endcsname{\def\PY@tc##1{\textcolor[rgb]{0.10,0.09,0.49}{##1}}}
\expandafter\def\csname PY@tok@mh\endcsname{\def\PY@tc##1{\textcolor[rgb]{0.40,0.40,0.40}{##1}}}
\expandafter\def\csname PY@tok@cs\endcsname{\let\PY@it=\textit\def\PY@tc##1{\textcolor[rgb]{0.25,0.50,0.50}{##1}}}
\expandafter\def\csname PY@tok@ge\endcsname{\let\PY@it=\textit}
\expandafter\def\csname PY@tok@vc\endcsname{\def\PY@tc##1{\textcolor[rgb]{0.10,0.09,0.49}{##1}}}
\expandafter\def\csname PY@tok@il\endcsname{\def\PY@tc##1{\textcolor[rgb]{0.40,0.40,0.40}{##1}}}
\expandafter\def\csname PY@tok@go\endcsname{\def\PY@tc##1{\textcolor[rgb]{0.53,0.53,0.53}{##1}}}
\expandafter\def\csname PY@tok@cp\endcsname{\def\PY@tc##1{\textcolor[rgb]{0.74,0.48,0.00}{##1}}}
\expandafter\def\csname PY@tok@gi\endcsname{\def\PY@tc##1{\textcolor[rgb]{0.00,0.63,0.00}{##1}}}
\expandafter\def\csname PY@tok@gh\endcsname{\let\PY@bf=\textbf\def\PY@tc##1{\textcolor[rgb]{0.00,0.00,0.50}{##1}}}
\expandafter\def\csname PY@tok@ni\endcsname{\let\PY@bf=\textbf\def\PY@tc##1{\textcolor[rgb]{0.60,0.60,0.60}{##1}}}
\expandafter\def\csname PY@tok@nl\endcsname{\def\PY@tc##1{\textcolor[rgb]{0.63,0.63,0.00}{##1}}}
\expandafter\def\csname PY@tok@nn\endcsname{\let\PY@bf=\textbf\def\PY@tc##1{\textcolor[rgb]{0.00,0.00,1.00}{##1}}}
\expandafter\def\csname PY@tok@no\endcsname{\def\PY@tc##1{\textcolor[rgb]{0.53,0.00,0.00}{##1}}}
\expandafter\def\csname PY@tok@na\endcsname{\def\PY@tc##1{\textcolor[rgb]{0.49,0.56,0.16}{##1}}}
\expandafter\def\csname PY@tok@nb\endcsname{\def\PY@tc##1{\textcolor[rgb]{0.00,0.50,0.00}{##1}}}
\expandafter\def\csname PY@tok@nc\endcsname{\let\PY@bf=\textbf\def\PY@tc##1{\textcolor[rgb]{0.00,0.00,1.00}{##1}}}
\expandafter\def\csname PY@tok@nd\endcsname{\def\PY@tc##1{\textcolor[rgb]{0.67,0.13,1.00}{##1}}}
\expandafter\def\csname PY@tok@ne\endcsname{\let\PY@bf=\textbf\def\PY@tc##1{\textcolor[rgb]{0.82,0.25,0.23}{##1}}}
\expandafter\def\csname PY@tok@nf\endcsname{\def\PY@tc##1{\textcolor[rgb]{0.00,0.00,1.00}{##1}}}
\expandafter\def\csname PY@tok@si\endcsname{\let\PY@bf=\textbf\def\PY@tc##1{\textcolor[rgb]{0.73,0.40,0.53}{##1}}}
\expandafter\def\csname PY@tok@s2\endcsname{\def\PY@tc##1{\textcolor[rgb]{0.73,0.13,0.13}{##1}}}
\expandafter\def\csname PY@tok@nt\endcsname{\let\PY@bf=\textbf\def\PY@tc##1{\textcolor[rgb]{0.00,0.50,0.00}{##1}}}
\expandafter\def\csname PY@tok@nv\endcsname{\def\PY@tc##1{\textcolor[rgb]{0.10,0.09,0.49}{##1}}}
\expandafter\def\csname PY@tok@s1\endcsname{\def\PY@tc##1{\textcolor[rgb]{0.73,0.13,0.13}{##1}}}
\expandafter\def\csname PY@tok@ch\endcsname{\let\PY@it=\textit\def\PY@tc##1{\textcolor[rgb]{0.25,0.50,0.50}{##1}}}
\expandafter\def\csname PY@tok@m\endcsname{\def\PY@tc##1{\textcolor[rgb]{0.40,0.40,0.40}{##1}}}
\expandafter\def\csname PY@tok@gp\endcsname{\let\PY@bf=\textbf\def\PY@tc##1{\textcolor[rgb]{0.00,0.00,0.50}{##1}}}
\expandafter\def\csname PY@tok@sh\endcsname{\def\PY@tc##1{\textcolor[rgb]{0.73,0.13,0.13}{##1}}}
\expandafter\def\csname PY@tok@ow\endcsname{\let\PY@bf=\textbf\def\PY@tc##1{\textcolor[rgb]{0.67,0.13,1.00}{##1}}}
\expandafter\def\csname PY@tok@sx\endcsname{\def\PY@tc##1{\textcolor[rgb]{0.00,0.50,0.00}{##1}}}
\expandafter\def\csname PY@tok@bp\endcsname{\def\PY@tc##1{\textcolor[rgb]{0.00,0.50,0.00}{##1}}}
\expandafter\def\csname PY@tok@c1\endcsname{\let\PY@it=\textit\def\PY@tc##1{\textcolor[rgb]{0.25,0.50,0.50}{##1}}}
\expandafter\def\csname PY@tok@o\endcsname{\def\PY@tc##1{\textcolor[rgb]{0.40,0.40,0.40}{##1}}}
\expandafter\def\csname PY@tok@kc\endcsname{\let\PY@bf=\textbf\def\PY@tc##1{\textcolor[rgb]{0.00,0.50,0.00}{##1}}}
\expandafter\def\csname PY@tok@c\endcsname{\let\PY@it=\textit\def\PY@tc##1{\textcolor[rgb]{0.25,0.50,0.50}{##1}}}
\expandafter\def\csname PY@tok@mf\endcsname{\def\PY@tc##1{\textcolor[rgb]{0.40,0.40,0.40}{##1}}}
\expandafter\def\csname PY@tok@err\endcsname{\def\PY@bc##1{\setlength{\fboxsep}{0pt}\fcolorbox[rgb]{1.00,0.00,0.00}{1,1,1}{\strut ##1}}}
\expandafter\def\csname PY@tok@mb\endcsname{\def\PY@tc##1{\textcolor[rgb]{0.40,0.40,0.40}{##1}}}
\expandafter\def\csname PY@tok@ss\endcsname{\def\PY@tc##1{\textcolor[rgb]{0.10,0.09,0.49}{##1}}}
\expandafter\def\csname PY@tok@sr\endcsname{\def\PY@tc##1{\textcolor[rgb]{0.73,0.40,0.53}{##1}}}
\expandafter\def\csname PY@tok@mo\endcsname{\def\PY@tc##1{\textcolor[rgb]{0.40,0.40,0.40}{##1}}}
\expandafter\def\csname PY@tok@kd\endcsname{\let\PY@bf=\textbf\def\PY@tc##1{\textcolor[rgb]{0.00,0.50,0.00}{##1}}}
\expandafter\def\csname PY@tok@mi\endcsname{\def\PY@tc##1{\textcolor[rgb]{0.40,0.40,0.40}{##1}}}
\expandafter\def\csname PY@tok@kn\endcsname{\let\PY@bf=\textbf\def\PY@tc##1{\textcolor[rgb]{0.00,0.50,0.00}{##1}}}
\expandafter\def\csname PY@tok@cpf\endcsname{\let\PY@it=\textit\def\PY@tc##1{\textcolor[rgb]{0.25,0.50,0.50}{##1}}}
\expandafter\def\csname PY@tok@kr\endcsname{\let\PY@bf=\textbf\def\PY@tc##1{\textcolor[rgb]{0.00,0.50,0.00}{##1}}}
\expandafter\def\csname PY@tok@s\endcsname{\def\PY@tc##1{\textcolor[rgb]{0.73,0.13,0.13}{##1}}}
\expandafter\def\csname PY@tok@kp\endcsname{\def\PY@tc##1{\textcolor[rgb]{0.00,0.50,0.00}{##1}}}
\expandafter\def\csname PY@tok@w\endcsname{\def\PY@tc##1{\textcolor[rgb]{0.73,0.73,0.73}{##1}}}
\expandafter\def\csname PY@tok@kt\endcsname{\def\PY@tc##1{\textcolor[rgb]{0.69,0.00,0.25}{##1}}}
\expandafter\def\csname PY@tok@sc\endcsname{\def\PY@tc##1{\textcolor[rgb]{0.73,0.13,0.13}{##1}}}
\expandafter\def\csname PY@tok@sb\endcsname{\def\PY@tc##1{\textcolor[rgb]{0.73,0.13,0.13}{##1}}}
\expandafter\def\csname PY@tok@k\endcsname{\let\PY@bf=\textbf\def\PY@tc##1{\textcolor[rgb]{0.00,0.50,0.00}{##1}}}
\expandafter\def\csname PY@tok@se\endcsname{\let\PY@bf=\textbf\def\PY@tc##1{\textcolor[rgb]{0.73,0.40,0.13}{##1}}}
\expandafter\def\csname PY@tok@sd\endcsname{\let\PY@it=\textit\def\PY@tc##1{\textcolor[rgb]{0.73,0.13,0.13}{##1}}}

\def\PYZbs{\char`\\}
\def\PYZus{\char`\_}
\def\PYZob{\char`\{}
\def\PYZcb{\char`\}}
\def\PYZca{\char`\^}
\def\PYZam{\char`\&}
\def\PYZlt{\char`\<}
\def\PYZgt{\char`\>}
\def\PYZsh{\char`\#}
\def\PYZpc{\char`\%}
\def\PYZdl{\char`\$}
\def\PYZhy{\char`\-}
\def\PYZsq{\char`\'}
\def\PYZdq{\char`\"}
\def\PYZti{\char`\~}
% for compatibility with earlier versions
\def\PYZat{@}
\def\PYZlb{[}
\def\PYZrb{]}
\makeatother


    % Exact colors from NB
    \definecolor{incolor}{rgb}{0.0, 0.0, 0.5}
    \definecolor{outcolor}{rgb}{0.545, 0.0, 0.0}



    
    % Prevent overflowing lines due to hard-to-break entities
    \sloppy 
    % Setup hyperref package
    \hypersetup{
      breaklinks=true,  % so long urls are correctly broken across lines
      colorlinks=true,
      urlcolor=blue,
      linkcolor=darkorange,
      citecolor=darkgreen,
      }
    % Slightly bigger margins than the latex defaults
    
    \geometry{verbose,tmargin=1in,bmargin=1in,lmargin=1in,rmargin=1in}
    
    

    \begin{document}
    
    
    \maketitle
    
    

    
    \section{Short introduction to
python}\label{short-introduction-to-python}

\subsection{Preliminary}\label{preliminary}

This notebook is part of the repository
\href{https://github.com/alfkjartan/systemanalys}{systemanalys}. You can
either clone this repository using \href{https://help.github.com/articles/set-up-git/}{\texttt{git}}, or you can download it
as a
\href{https://github.com/alfkjartan/systemanalys/archive/master.zip}{zip-file}.

\subsection{Installation}\label{installation}

\subsubsection{Anaconda}\label{anaconda}

In this course we recomment to use the
\href{https://www.continuum.io/downloads}{Anaconda} python distribution.
Make sure to download the Python3 version. The Anaconda distribution
contains by default many of the most common python modules for
scientific computing. However, it does not come with
\href{https://simpy.readthedocs.io/en/latest/}{SimPy}. To install this,
open the Anaconda prompt (Start -\textgreater{} All Programs
-\textgreater{} Anaconda -\textgreater{} Anaconda prompt) and write

\begin{quote}
\texttt{\textgreater{}\ pip\ install\ simpy}
\end{quote}

You can manage your Anaconda installation and launch programs from the
Anaconda Navigator (Start -\textgreater{} All Programs -\textgreater{}
Anaconda -\textgreater{} Anaconda Navigator). The anaconda distribution
does come with Jupyter, which is a program that lets you work with
notebooks (code and documentation interwoven) in your favourite browser.
Wonderful to work with!

\subsubsection{Jupyter}\label{jupyter}

Writing and running the simulation will be done using
\href{https://jupyter.readthedocs.io/en/latest/index.html}{Jupyter}
notebooks. This getting-started-guide is written as a Jupyter notebook.
You may be reading this and running the command in jupyter. Or if you
are reading the pdf-version you are encouraged to jump over to the
Jupyter version right away. You can launch Jupyter notebook from the
Anaconda Navigator. This will open a new tab in your default web
browser. It shows the current working directory from which you can
create folders, files or notebooks. Navigate to the folder containing
this notebook (\texttt{systemanalys/doc}) and open the notebook
Introduction to Python.

    \subsection{Python variables and
datatypes}\label{python-variables-and-datatypes}

To do something useful, we need to define \emph{variables} to hold
\emph{data}. In python the \emph{type} of the variable is not specified,
so a variable name can hold any kind of data. This is called ``weak
typing''. The most common basic data types are

    numbers

    \begin{Verbatim}[commandchars=\\\{\}]
{\color{incolor}In [{\color{incolor}9}]:} \PY{n}{a} \PY{o}{=} \PY{l+m+mi}{42}
\end{Verbatim}

    strings

    \begin{Verbatim}[commandchars=\\\{\}]
{\color{incolor}In [{\color{incolor}10}]:} \PY{n}{s1} \PY{o}{=} \PY{l+s+s1}{\PYZsq{}}\PY{l+s+s1}{system}\PY{l+s+s1}{\PYZsq{}}
         \PY{n}{s2} \PY{o}{=} \PY{l+s+s2}{\PYZdq{}}\PY{l+s+s2}{system}\PY{l+s+s2}{\PYZdq{}}
\end{Verbatim}

    lists

    \begin{Verbatim}[commandchars=\\\{\}]
{\color{incolor}In [{\color{incolor}11}]:} \PY{n}{b} \PY{o}{=} \PY{p}{[}\PY{l+s+s1}{\PYZsq{}}\PY{l+s+s1}{hej}\PY{l+s+s1}{\PYZsq{}}\PY{p}{,} \PY{n}{s1}\PY{p}{,} \PY{n}{a}\PY{p}{]}
\end{Verbatim}

    In the assignments above, this is what happens in the computer: Memory
space is allocated to hold the data (what is on the right hand side of
the equal sign), and a variable name is created to reference that data.
Also, the variable name is added to the list of current variables. When
one variable is assigned to another, the data is not copied (unless it
is a simple data type such as a float or integer). Instead we end up
with two variables referencing the same piece of data. For instance

    \begin{Verbatim}[commandchars=\\\{\}]
{\color{incolor}In [{\color{incolor}12}]:} \PY{n}{c} \PY{o}{=} \PY{n}{b}
         \PY{n}{c}
\end{Verbatim}

            \begin{Verbatim}[commandchars=\\\{\}]
{\color{outcolor}Out[{\color{outcolor}12}]:} ['hej', 'system', 42]
\end{Verbatim}
        
    \begin{Verbatim}[commandchars=\\\{\}]
{\color{incolor}In [{\color{incolor}13}]:} \PY{n}{c}\PY{p}{[}\PY{o}{\PYZhy{}}\PY{l+m+mi}{1}\PY{p}{]} \PY{o}{=} \PY{l+m+mi}{24}
         \PY{n+nb}{print}\PY{p}{(}\PY{n}{c}\PY{p}{)}
         \PY{n+nb}{print}\PY{p}{(}\PY{n}{b}\PY{p}{)}
\end{Verbatim}

    \begin{Verbatim}[commandchars=\\\{\}]
['hej', 'system', 24]
['hej', 'system', 24]
    \end{Verbatim}

    Note the command \texttt{c{[}-1{]}\ =\ 24}. Negative indices in python
is a convenient way of accessing elements starting from the end of the
list. Note also that the command is (re)assigning the last element of
\texttt{c} to the value 24, leaving the other elements unchanged. The
data referenced by \texttt{c} is modified, and the variable \texttt{b}
which refers to the same data will now show the same change. This is
fundamentally different from writing

    \begin{Verbatim}[commandchars=\\\{\}]
{\color{incolor}In [{\color{incolor}17}]:} \PY{n}{c} \PY{o}{=} \PY{p}{[}\PY{p}{]}
         \PY{n}{b}
\end{Verbatim}

            \begin{Verbatim}[commandchars=\\\{\}]
{\color{outcolor}Out[{\color{outcolor}17}]:} ['hej', 'system', 24]
\end{Verbatim}
        
    Above, an empty list is created and the variable \texttt{c} is then
assigned to this empty list instead of whatever it was assigned to
earlier. The two variables \texttt{c} and \texttt{b} no longer reference
the same data, and the operation leaves \texttt{b} entirely unchanged.

    \subsection{If, for and while}\label{if-for-and-while}

The \texttt{if} construct is used whenever certain lines of code should
be executed only when some condition holds. For instance

    \begin{Verbatim}[commandchars=\\\{\}]
{\color{incolor}In [{\color{incolor}18}]:} \PY{k}{if} \PY{p}{(}\PY{n}{c} \PY{o}{==} \PY{p}{[}\PY{p}{]}\PY{p}{)}\PY{p}{:}
             \PY{n+nb}{print}\PY{p}{(}\PY{l+s+s2}{\PYZdq{}}\PY{l+s+s2}{c is empty}\PY{l+s+s2}{\PYZdq{}}\PY{p}{)}
\end{Verbatim}

    \begin{Verbatim}[commandchars=\\\{\}]
c is empty
    \end{Verbatim}

    The block of code (lines of code) to be executed if the \texttt{if}
statement is true is identified by being further \emph{indented} than
the line starting with \texttt{if}. There are no curly brackets or
keywords marking the start and end of the block. The first line that are
equally or less indented than the \texttt{if} line ends the block of
code to be executed. In the example above, this line is empty.

To repeat some piece of code a known number of times, use a \texttt{for}
loop:

    \begin{Verbatim}[commandchars=\\\{\}]
{\color{incolor}In [{\color{incolor}20}]:} \PY{k}{for} \PY{n}{i} \PY{o+ow}{in} \PY{n+nb}{range}\PY{p}{(}\PY{n+nb}{len}\PY{p}{(}\PY{n}{b}\PY{p}{)}\PY{p}{)}\PY{p}{:}
             \PY{n+nb}{print}\PY{p}{(}\PY{n}{b}\PY{p}{[}\PY{n}{i}\PY{p}{]}\PY{p}{)}
\end{Verbatim}

    \begin{Verbatim}[commandchars=\\\{\}]
hej
system
24
    \end{Verbatim}

    The function \texttt{len()} returns the length of the list, and the
function \texttt{range(n)} returns a list of integers starting at 0 and
ending at \(n-1\). It may seem a bit unnecessary to create a list of
integers to use as indices into \texttt{b}. Sometimes the index is
actually needed, but more often we loop over the elements more
conveniently

    \begin{Verbatim}[commandchars=\\\{\}]
{\color{incolor}In [{\color{incolor}21}]:} \PY{k}{for} \PY{n}{e} \PY{o+ow}{in} \PY{n}{b}\PY{p}{:}
             \PY{n+nb}{print}\PY{p}{(}\PY{n}{e}\PY{p}{)}
\end{Verbatim}

    \begin{Verbatim}[commandchars=\\\{\}]
hej
system
24
    \end{Verbatim}

    A \texttt{while} loop is useful to repeat a number of lines for as long
as some condition is true

    \begin{Verbatim}[commandchars=\\\{\}]
{\color{incolor}In [{\color{incolor}22}]:} \PY{n}{x} \PY{o}{=} \PY{l+m+mi}{7}
         \PY{k}{while} \PY{n}{x} \PY{o}{\PYZlt{}} \PY{l+m+mi}{1000}\PY{p}{:}
             \PY{n}{x} \PY{o}{+}\PY{o}{=} \PY{l+m+mi}{7}
         \PY{n}{x}
\end{Verbatim}

            \begin{Verbatim}[commandchars=\\\{\}]
{\color{outcolor}Out[{\color{outcolor}22}]:} 1001
\end{Verbatim}
        
    \texttt{while} can also be used to create an eternal loop

    \begin{Verbatim}[commandchars=\\\{\}]
{\color{incolor}In [{\color{incolor} }]:} \PY{k}{while} \PY{k+kc}{True}\PY{p}{:}
            \PY{n}{x} \PY{o}{*}\PY{o}{=} \PY{n}{x}
        \PY{n}{x}
\end{Verbatim}

    To interrupt the neverending execution of the above code, look to the
Jupyter menu near the top of the browser page and choose Kernel
-\textgreater{} Interrupt

    \subsection{Modules}\label{modules}

So far we have entered the code directly into the command window. More
commonly, one writes code in text files which are then read and executed
by the interpreter. A python file (with ending \texttt{.py}) containing
code is referred to as a \emph{module}. At start-up, some of these
modules are loaded automatically and are available for use in our own
code. Other modules must be explicitly imported before the code can be
accessed. There is, for instance, a \texttt{math} module containing a
number of mathematical functions, and `numpy' for numerical computations
and linear algebra. To access functions from these modules we must do

    \begin{Verbatim}[commandchars=\\\{\}]
{\color{incolor}In [{\color{incolor}26}]:} \PY{k+kn}{import} \PY{n+nn}{math}
         \PY{k+kn}{import} \PY{n+nn}{numpy} \PY{k}{as} \PY{n+nn}{np}
\end{Verbatim}

    then we can use functions such as

    \begin{Verbatim}[commandchars=\\\{\}]
{\color{incolor}In [{\color{incolor}27}]:} \PY{n+nb}{print} \PY{p}{(}\PY{n}{math}\PY{o}{.}\PY{n}{sin}\PY{p}{(}\PY{n}{math}\PY{o}{.}\PY{n}{pi}\PY{p}{)}\PY{p}{)}
         \PY{n+nb}{print} \PY{p}{(}\PY{n}{np}\PY{o}{.}\PY{n}{sin}\PY{p}{(}\PY{n}{np}\PY{o}{.}\PY{n}{pi}\PY{o}{/}\PY{l+m+mi}{2}\PY{p}{)}\PY{p}{)}
\end{Verbatim}

    \begin{Verbatim}[commandchars=\\\{\}]
1.2246467991473532e-16
1.0
    \end{Verbatim}

    other modules of interest include
\href{http://docs.scipy.org/doc/scipy/reference/}{SciPy} for scientific
computing, \href{http://matplotlib.org/}{matplotlib} for plotting and
\href{http://pandas.pydata.org/}{pandas} for data analysis. For
simulations of discrete event simulations we will use the
\href{https://simpy.readthedocs.io/en/latest/}{SimPy} module.

    \subsection{Functions}\label{functions}

Functions are fundamental constructs in programming languages. They make
code much easier to understand and maintain by separating and
encapsulating operations. Consider this simple function that swops the
first and last element of a list:

    \begin{Verbatim}[commandchars=\\\{\}]
{\color{incolor}In [{\color{incolor}32}]:} \PY{k}{def} \PY{n+nf}{swop}\PY{p}{(}\PY{n}{alist}\PY{p}{)}\PY{p}{:}
            \PY{n}{slask} \PY{o}{=} \PY{n}{alist}\PY{p}{[}\PY{l+m+mi}{0}\PY{p}{]}
            \PY{n}{alist}\PY{p}{[}\PY{l+m+mi}{0}\PY{p}{]} \PY{o}{=} \PY{n}{alist}\PY{p}{[}\PY{o}{\PYZhy{}}\PY{l+m+mi}{1}\PY{p}{]}
            \PY{n}{alist}\PY{p}{[}\PY{o}{\PYZhy{}}\PY{l+m+mi}{1}\PY{p}{]} \PY{o}{=} \PY{n}{slask}
             \PY{c+c1}{\PYZsh{} Think about cases where the function will fail and try to figure out how to make it more robust}
         
         \PY{n+nb}{print}\PY{p}{(}\PY{n}{b}\PY{p}{)}
         \PY{n}{swop}\PY{p}{(}\PY{n}{b}\PY{p}{)}
         \PY{n+nb}{print}\PY{p}{(}\PY{n}{b}\PY{p}{)}
\end{Verbatim}

    \begin{Verbatim}[commandchars=\\\{\}]
[24, 'system', 'hej']
['hej', 'system', 24]
    \end{Verbatim}

    The function above did some operations on the input argument, but did
not return any value. This is of course possible also:

    \begin{Verbatim}[commandchars=\\\{\}]
{\color{incolor}In [{\color{incolor}36}]:} \PY{k}{def} \PY{n+nf}{middle}\PY{p}{(}\PY{n}{alist}\PY{p}{)}\PY{p}{:}
           \PY{l+s+sd}{\PYZsq{}\PYZsq{}\PYZsq{} Will return the element in the middle of the list. }
         \PY{l+s+sd}{      If the length is even, it will return the last element before the middle. \PYZsq{}\PYZsq{}\PYZsq{}}
           \PY{n}{ind} \PY{o}{=} \PY{n+nb}{int}\PY{p}{(}\PY{n+nb}{len}\PY{p}{(}\PY{n}{alist}\PY{p}{)}\PY{o}{/}\PY{l+m+mi}{2} \PY{o}{\PYZhy{}}\PY{l+m+mi}{1}\PY{p}{)} \PY{c+c1}{\PYZsh{} since indexing starts at 0}
           \PY{k}{return} \PY{n}{alist}\PY{p}{[}\PY{n}{ind}\PY{p}{]}
         
         \PY{n}{middle}\PY{p}{(}\PY{n}{b}\PY{p}{)}
\end{Verbatim}

            \begin{Verbatim}[commandchars=\\\{\}]
{\color{outcolor}Out[{\color{outcolor}36}]:} 'hej'
\end{Verbatim}
        
    \subsection{Creating your own
datatypes}\label{creating-your-own-datatypes}

Basically all modern programming languages allows you to create your own
datatypes based on the fundamental types that the language provides.
Assume that we would like to create a datatype \texttt{Customer}. We
want to use it as this:

    \begin{Verbatim}[commandchars=\\\{\}]
{\color{incolor}In [{\color{incolor} }]:} \PY{n}{c1} \PY{o}{=} \PY{n}{Customer}\PY{p}{(}\PY{l+s+s2}{\PYZdq{}}\PY{l+s+s2}{Pontus}\PY{l+s+s2}{\PYZdq{}}\PY{p}{,} \PY{n}{prio}\PY{o}{=}\PY{l+m+mi}{1}\PY{p}{)}
        \PY{n}{c2} \PY{o}{=} \PY{n}{Curstomer}\PY{p}{(}\PY{l+s+s2}{\PYZdq{}}\PY{l+s+s2}{Elisa}\PY{l+s+s2}{\PYZdq{}}\PY{p}{,} \PY{n}{prio}\PY{o}{=}\PY{l+m+mi}{3}\PY{p}{)}
        \PY{n}{queue} \PY{o}{=} \PY{p}{[}\PY{n}{c1}\PY{p}{,} \PY{n}{c2}\PY{p}{]}
\end{Verbatim}

    Which gives us a short queue containing two customers. The answer is to
define a \emph{class} \texttt{Customer} from which we can instantiate
(generate) any number of \emph{objects}. The values \texttt{name} and
\texttt{prio} are \emph{attributes} of the object:

    \begin{Verbatim}[commandchars=\\\{\}]
{\color{incolor}In [{\color{incolor}45}]:} \PY{k}{class} \PY{n+nc}{Customer}\PY{p}{:}
           \PY{k}{def} \PY{n+nf}{\PYZus{}\PYZus{}init\PYZus{}\PYZus{}}\PY{p}{(}\PY{n+nb+bp}{self}\PY{p}{,} \PY{n}{name}\PY{p}{,} \PY{n}{prio}\PY{o}{=}\PY{l+m+mi}{1}\PY{p}{)}\PY{p}{:}
              \PY{n+nb+bp}{self}\PY{o}{.}\PY{n}{name} \PY{o}{=} \PY{n}{name}
              \PY{n+nb+bp}{self}\PY{o}{.}\PY{n}{prio} \PY{o}{=} \PY{n}{prio}
         
         \PY{n}{c1} \PY{o}{=} \PY{n}{Customer}\PY{p}{(}\PY{l+s+s2}{\PYZdq{}}\PY{l+s+s2}{Pontus}\PY{l+s+s2}{\PYZdq{}}\PY{p}{,} \PY{n}{prio}\PY{o}{=}\PY{l+m+mi}{1}\PY{p}{)}
         \PY{n}{c2} \PY{o}{=} \PY{n}{Customer}\PY{p}{(}\PY{l+s+s2}{\PYZdq{}}\PY{l+s+s2}{Elisa}\PY{l+s+s2}{\PYZdq{}}\PY{p}{,} \PY{n}{prio}\PY{o}{=}\PY{l+m+mi}{3}\PY{p}{)}
         \PY{n}{queue} \PY{o}{=} \PY{p}{[}\PY{n}{c1}\PY{p}{,} \PY{n}{c2}\PY{p}{]}
         \PY{n}{queue}
\end{Verbatim}

            \begin{Verbatim}[commandchars=\\\{\}]
{\color{outcolor}Out[{\color{outcolor}45}]:} [<\_\_main\_\_.Customer at 0x7fb9345a1940>, <\_\_main\_\_.Customer at 0x7fb9345a15f8>]
\end{Verbatim}
        
    The function \texttt{\_\_init()\_\_} is called the \emph{constructor}.
It is called when a new object is created. The first argument,
\texttt{self}, is a reference to the actual object under construction.
We see from the code that the constructor takes two arguments: the name
and the priority of the customer. The priority has a default value of 1,
which is used if the customer object is created with one argument only.
We may check the name and priority of the customers as

    \begin{Verbatim}[commandchars=\\\{\}]
{\color{incolor}In [{\color{incolor}46}]:} \PY{k}{for} \PY{n}{c} \PY{o+ow}{in} \PY{n}{queue}\PY{p}{:}
           \PY{n+nb}{print}\PY{p}{(} \PY{l+s+s2}{\PYZdq{}}\PY{l+s+s2}{Customer }\PY{l+s+si}{\PYZpc{}s}\PY{l+s+s2}{ has priority }\PY{l+s+si}{\PYZpc{}d}\PY{l+s+s2}{\PYZdq{}} \PY{o}{\PYZpc{}} \PY{p}{(}\PY{n}{c}\PY{o}{.}\PY{n}{name}\PY{p}{,} \PY{n}{c}\PY{o}{.}\PY{n}{prio}\PY{p}{)} \PY{p}{)}
\end{Verbatim}

    \begin{Verbatim}[commandchars=\\\{\}]
Customer Pontus has priority 1
Customer Elisa has priority 3
    \end{Verbatim}

    The somewhat complicated syntax on the \texttt{print} line is actually a
quite common way (many programming languages has it) of formatting
output that contains numbers. Within the string, the
percentage-expressions are placeholders for numbers or strings. The
occurance of the placeholders are matched with the elements in the
\emph{tuple} following the string (the tuple is the list within
parantheses separated from the string by the percentage symbol). A tuple
is a fundamental datatype in python. It is a list whose elements cannot
be changed after it is constructed.

One important advantage of objects and object-oriented programming is
that functions written to work on the objects can be associated with the
object itself. These functions are called \emph{methods} or \emph{member
functions}. If we have an object, say a \texttt{list} object, we can
print a list of the methods and attributes for that object with the help
of the \texttt{dir} function:

    \begin{Verbatim}[commandchars=\\\{\}]
{\color{incolor}In [{\color{incolor}47}]:} \PY{n}{names} \PY{o}{=} \PY{p}{[}\PY{l+s+s1}{\PYZsq{}}\PY{l+s+s1}{Evelyn}\PY{l+s+s1}{\PYZsq{}}\PY{p}{,} \PY{l+s+s1}{\PYZsq{}}\PY{l+s+s1}{Eirik}\PY{l+s+s1}{\PYZsq{}}\PY{p}{]}
         \PY{n+nb}{dir}\PY{p}{(}\PY{n}{names}\PY{p}{)}
\end{Verbatim}

            \begin{Verbatim}[commandchars=\\\{\}]
{\color{outcolor}Out[{\color{outcolor}47}]:} ['\_\_add\_\_',
          '\_\_class\_\_',
          '\_\_contains\_\_',
          '\_\_delattr\_\_',
          '\_\_delitem\_\_',
          '\_\_dir\_\_',
          '\_\_doc\_\_',
          '\_\_eq\_\_',
          '\_\_format\_\_',
          '\_\_ge\_\_',
          '\_\_getattribute\_\_',
          '\_\_getitem\_\_',
          '\_\_gt\_\_',
          '\_\_hash\_\_',
          '\_\_iadd\_\_',
          '\_\_imul\_\_',
          '\_\_init\_\_',
          '\_\_init\_subclass\_\_',
          '\_\_iter\_\_',
          '\_\_le\_\_',
          '\_\_len\_\_',
          '\_\_lt\_\_',
          '\_\_mul\_\_',
          '\_\_ne\_\_',
          '\_\_new\_\_',
          '\_\_reduce\_\_',
          '\_\_reduce\_ex\_\_',
          '\_\_repr\_\_',
          '\_\_reversed\_\_',
          '\_\_rmul\_\_',
          '\_\_setattr\_\_',
          '\_\_setitem\_\_',
          '\_\_sizeof\_\_',
          '\_\_str\_\_',
          '\_\_subclasshook\_\_',
          'append',
          'clear',
          'copy',
          'count',
          'extend',
          'index',
          'insert',
          'pop',
          'remove',
          'reverse',
          'sort']
\end{Verbatim}
        
    You can use the \texttt{help()} function to get help on specific
methods. For instance

    \begin{Verbatim}[commandchars=\\\{\}]
{\color{incolor}In [{\color{incolor}48}]:} \PY{n}{help}\PY{p}{(}\PY{n}{names}\PY{o}{.}\PY{n}{sort}\PY{p}{)}
\end{Verbatim}

    \begin{Verbatim}[commandchars=\\\{\}]
Help on built-in function sort:

sort({\ldots}) method of builtins.list instance
    L.sort(key=None, reverse=False) -> None -- stable sort *IN PLACE*
    \end{Verbatim}

    This help string tells us that the \texttt{sort()} method will sort the
list in place, i.e.~it will be modified by this method. It will try to
sort the list depending on the types of the elements. So sorting a list
of strings is

    \begin{Verbatim}[commandchars=\\\{\}]
{\color{incolor}In [{\color{incolor}51}]:} \PY{n}{names} \PY{o}{=} \PY{p}{[}\PY{l+s+s2}{\PYZdq{}}\PY{l+s+s2}{Evelyn}\PY{l+s+s2}{\PYZdq{}}\PY{p}{,} \PY{l+s+s2}{\PYZdq{}}\PY{l+s+s2}{Elisa}\PY{l+s+s2}{\PYZdq{}}\PY{p}{,} \PY{l+s+s2}{\PYZdq{}}\PY{l+s+s2}{Eirik}\PY{l+s+s2}{\PYZdq{}}\PY{p}{,} \PY{l+s+s2}{\PYZdq{}}\PY{l+s+s2}{Emilio}\PY{l+s+s2}{\PYZdq{}}\PY{p}{]}
         \PY{n+nb}{print}\PY{p}{(}\PY{n}{names}\PY{p}{)}
         \PY{n}{names}\PY{o}{.}\PY{n}{sort}\PY{p}{(}\PY{p}{)}
         \PY{n}{names}
\end{Verbatim}

    \begin{Verbatim}[commandchars=\\\{\}]
['Evelyn', 'Elisa', 'Eirik', 'Emilio']
    \end{Verbatim}

            \begin{Verbatim}[commandchars=\\\{\}]
{\color{outcolor}Out[{\color{outcolor}51}]:} ['Eirik', 'Elisa', 'Emilio', 'Evelyn']
\end{Verbatim}
        
    \subsection{Generators}\label{generators}

\emph{Generators} is a powerful concept in python. It is used to
implement the processes in the pseudo-parallell / process-based
simulations in SimPy. You should think of generators as functions which
generateas and returns elements in a sequence. Let's look at a simple
example which generates a sequence of odd numbers:

    \begin{Verbatim}[commandchars=\\\{\}]
{\color{incolor}In [{\color{incolor}54}]:} \PY{k}{def} \PY{n+nf}{oddnumbers}\PY{p}{(}\PY{p}{)}\PY{p}{:}
           \PY{n}{x} \PY{o}{=} \PY{l+m+mi}{1}
           \PY{k}{while} \PY{k+kc}{True}\PY{p}{:}
             \PY{k}{yield} \PY{n}{x}
             \PY{n}{x} \PY{o}{=} \PY{n}{x}\PY{o}{+}\PY{l+m+mi}{2}
         
         \PY{n}{og} \PY{o}{=} \PY{n}{oddnumbers}\PY{p}{(}\PY{p}{)} \PY{c+c1}{\PYZsh{} This creates the generator}
         
         \PY{n+nb}{print}\PY{p}{(} \PY{n+nb}{next}\PY{p}{(}\PY{n}{og}\PY{p}{)} \PY{p}{)} \PY{c+c1}{\PYZsh{} Prints 1}
         \PY{n+nb}{print}\PY{p}{(} \PY{n+nb}{next}\PY{p}{(}\PY{n}{og}\PY{p}{)} \PY{p}{)} \PY{c+c1}{\PYZsh{} Prints 3}
         
         \PY{k}{for} \PY{n}{i} \PY{o+ow}{in} \PY{n+nb}{range}\PY{p}{(}\PY{l+m+mi}{4}\PY{p}{)}\PY{p}{:} \PY{c+c1}{\PYZsh{} The odd numbers 5, 7, 9, 11}
           \PY{n+nb}{print}\PY{p}{(} \PY{n+nb}{next}\PY{p}{(}\PY{n}{og}\PY{p}{)} \PY{p}{)}
\end{Verbatim}

    \begin{Verbatim}[commandchars=\\\{\}]
1
3
5
7
9
11
    \end{Verbatim}

    Python recognizes a definition of a generator by the existence of the
keyword \texttt{yield}. In the example, a generator object is created
\texttt{og\ =\ oddnumbers()}, and then the function \texttt{next(og)} is
called on the generator object. The first call to \texttt{next()} causes
the code in the definition of the generator to be executed from the top
to the first occurance of \texttt{yield}. The statment \texttt{yield} is
similar to a return statement in a regular function. However, in the
next call to \texttt{next()}, the execution starts at the first line
after the \texttt{yield} statement that caused the previous return to
the caller. The execution goes on to the next occurance of
\texttt{yield}. The local variables in the definition of the generator
retains there values between calls to \texttt{next()}.

If a call to \texttt{next()} should not reach a \texttt{yield}
statement, but instead reach the end of the definition of the generator,
then an exception is raised which tells the caller to stop iterating
over the generator: There are no more elements in the sequence. The odd
numbers are of course an infinite sequence, so it should be possible (in
theory) to generate increasingly large odd numbers by calling
\texttt{next(og)} forever.

    \subsubsection{A classical example: the Fibonacci
numbers}\label{a-classical-example-the-fibonacci-numbers}

The Fibonacci number series is a sequence of integers defined by the
difference equation \[ F(n) = F(n-1) + F(n-2) \] together with the
starting conditions \(F(0) = 0\), \(F(1) = 1\). Leonardo Fibonacci used
the difference equation as a simple model of the dynamics in a
population of rabbits. In the model there are no deaths, so at each time
step, the number of rabbits is equal to the number of rabbits in the
previous time step plus the number of rabbits born since the last time
step. Each rabbit has exactly one offspring in each time step, except
for an initial newborn-period of one time step. Hence the number of
rabbits in the last time step that are mature enough to reproduce equals
the size of the population two time steps back in time.

A standard schoolbook implementation of the fibonacci numbers would use
recursion to compute the numbers:

    \begin{Verbatim}[commandchars=\\\{\}]
{\color{incolor}In [{\color{incolor}58}]:} \PY{k}{def} \PY{n+nf}{fib\PYZus{}rec}\PY{p}{(}\PY{n}{n}\PY{p}{)}\PY{p}{:}
             \PY{k}{if} \PY{n}{n}\PY{o}{==}\PY{l+m+mi}{1} \PY{o+ow}{or} \PY{n}{n}\PY{o}{==}\PY{l+m+mi}{0}\PY{p}{:}
                 \PY{k}{return} \PY{n}{n}
             \PY{k}{return} \PY{n}{fib\PYZus{}rec}\PY{p}{(}\PY{n}{n}\PY{o}{\PYZhy{}}\PY{l+m+mi}{1}\PY{p}{)}\PY{o}{+}\PY{n}{fib\PYZus{}rec}\PY{p}{(}\PY{n}{n}\PY{o}{\PYZhy{}}\PY{l+m+mi}{2}\PY{p}{)}
         
         \PY{n}{fib\PYZus{}rec}\PY{p}{(}\PY{l+m+mi}{7}\PY{p}{)}
\end{Verbatim}

            \begin{Verbatim}[commandchars=\\\{\}]
{\color{outcolor}Out[{\color{outcolor}58}]:} 13
\end{Verbatim}
        
    The implementation looks kind of nice, but this type of recursive
implementation is not a good idea in a modern high-level language such
as python. The reason being that function calls are quite expensive as
compared to C. Each call to \texttt{fib\_rec()} causes two new calls to
the function. Also, values are computed twice.

The sequence is much more efficiently implemented with a generator

    \begin{Verbatim}[commandchars=\\\{\}]
{\color{incolor}In [{\color{incolor}60}]:} \PY{k}{def} \PY{n+nf}{fib}\PY{p}{(}\PY{p}{)}\PY{p}{:}
             \PY{l+s+sd}{\PYZdq{}\PYZdq{}\PYZdq{} Generates the sequence if fibonacci numbers \PYZdq{}\PYZdq{}\PYZdq{}}
             \PY{n}{fmin1} \PY{o}{=} \PY{l+m+mi}{0}
             \PY{n}{fmin2} \PY{o}{=} \PY{l+m+mi}{1}
             \PY{k}{while} \PY{k+kc}{True}\PY{p}{:}
                 \PY{n}{f} \PY{o}{=} \PY{n}{fmin1}\PY{o}{+}\PY{n}{fmin2}
                 \PY{k}{yield} \PY{n}{f}
                 \PY{c+c1}{\PYZsh{} Update}
                 \PY{p}{(}\PY{n}{fmin2}\PY{p}{,} \PY{n}{fmin1}\PY{p}{)} \PY{o}{=} \PY{p}{(}\PY{n}{fmin1}\PY{p}{,} \PY{n}{f}\PY{p}{)}
         
         \PY{n}{fg} \PY{o}{=} \PY{n}{fib}\PY{p}{(}\PY{p}{)}
         \PY{p}{[}\PY{n+nb}{next}\PY{p}{(}\PY{n}{fg}\PY{p}{)} \PY{k}{for} \PY{n}{i} \PY{o+ow}{in} \PY{n+nb}{range}\PY{p}{(}\PY{l+m+mi}{7}\PY{p}{)}\PY{p}{]}
\end{Verbatim}

            \begin{Verbatim}[commandchars=\\\{\}]
{\color{outcolor}Out[{\color{outcolor}60}]:} [1, 1, 2, 3, 5, 8, 13]
\end{Verbatim}
        
    The line \texttt{{[}next(fg)\ for\ i\ in\ range(7){]}} is an example of
\emph{list comprehension}. A list is generated from the return values of
the function call \texttt{next(fg)}. This is done 7 times.

Timing the two implementations of the fibonacci numbers show clearly the
difference in efficiency.

    \begin{Verbatim}[commandchars=\\\{\}]
{\color{incolor}In [{\color{incolor}66}]:} \PY{k+kn}{import} \PY{n+nn}{time}
         \PY{n}{t} \PY{o}{=} \PY{n}{time}\PY{o}{.}\PY{n}{time}\PY{p}{(}\PY{p}{)}
         \PY{n}{fg} \PY{o}{=} \PY{n}{fib}\PY{p}{(}\PY{p}{)}
         \PY{n}{fibonacciSequence} \PY{o}{=} \PY{p}{[}\PY{n+nb}{next}\PY{p}{(}\PY{n}{fg}\PY{p}{)} \PY{k}{for} \PY{n}{i} \PY{o+ow}{in} \PY{n+nb}{range}\PY{p}{(}\PY{l+m+mi}{30}\PY{p}{)}\PY{p}{]}
         \PY{n+nb}{print}\PY{p}{(} \PY{n}{time}\PY{o}{.}\PY{n}{time}\PY{p}{(}\PY{p}{)} \PY{o}{\PYZhy{}} \PY{n}{t}\PY{p}{)}
         \PY{n}{t} \PY{o}{=} \PY{n}{time}\PY{o}{.}\PY{n}{time}\PY{p}{(}\PY{p}{)}
         \PY{n}{fib\PYZus{}rec}\PY{p}{(}\PY{l+m+mi}{30}\PY{p}{)}
         \PY{n+nb}{print}\PY{p}{(} \PY{n}{time}\PY{o}{.}\PY{n}{time}\PY{p}{(}\PY{p}{)} \PY{o}{\PYZhy{}} \PY{n}{t}\PY{p}{)}
\end{Verbatim}

    \begin{Verbatim}[commandchars=\\\{\}]
0.00012254714965820312
0.4177854061126709
    \end{Verbatim}

    \subsection{Random number generation}\label{random-number-generation}

Discrete event systems are almost always stochastic systems. To build a
model and simulate it, we need sequences of random numbers. A truly
random sequence of numbers is actually very difficult to generate.
Instead, the computer will generate a completely deterministic sequence
of numbers, but whose properties are close to that of a sequence of
random numbers. The numbers are therefore called pseudo-random.

Python, as well as most other programming languages used in scientific
computing, has a built-in pseudo-number generator. We don't have to
implement this ourselves. The pseudo-number generator is part of the
\href{https://docs.python.org/2/library/random.html}{random} module. It
is used as follows

    \begin{Verbatim}[commandchars=\\\{\}]
{\color{incolor}In [{\color{incolor}68}]:} \PY{k+kn}{import} \PY{n+nn}{random}
         \PY{n}{random}\PY{o}{.}\PY{n}{seed}\PY{p}{(}\PY{l+m+mi}{1315}\PY{p}{)}
         \PY{n+nb}{print}\PY{p}{(} \PY{p}{[}\PY{n}{random}\PY{o}{.}\PY{n}{gauss}\PY{p}{(}\PY{n}{mu}\PY{o}{=}\PY{l+m+mf}{1.0}\PY{p}{,} \PY{n}{sigma}\PY{o}{=}\PY{l+m+mf}{0.5}\PY{p}{)} \PY{k}{for} \PY{n}{i} \PY{o+ow}{in} \PY{n+nb}{range}\PY{p}{(}\PY{l+m+mi}{3}\PY{p}{)}\PY{p}{]} \PY{p}{)}
         \PY{n}{random}\PY{o}{.}\PY{n}{seed}\PY{p}{(}\PY{l+m+mi}{1314}\PY{p}{)}
         \PY{n+nb}{print}\PY{p}{(} \PY{p}{[}\PY{n}{random}\PY{o}{.}\PY{n}{gauss}\PY{p}{(}\PY{n}{mu}\PY{o}{=}\PY{l+m+mf}{1.0}\PY{p}{,} \PY{n}{sigma}\PY{o}{=}\PY{l+m+mf}{0.5}\PY{p}{)} \PY{k}{for} \PY{n}{i} \PY{o+ow}{in} \PY{n+nb}{range}\PY{p}{(}\PY{l+m+mi}{3}\PY{p}{)}\PY{p}{]} \PY{p}{)}
         \PY{n}{random}\PY{o}{.}\PY{n}{seed}\PY{p}{(}\PY{l+m+mi}{1315}\PY{p}{)}
         \PY{n+nb}{print}\PY{p}{(} \PY{p}{[}\PY{n}{random}\PY{o}{.}\PY{n}{gauss}\PY{p}{(}\PY{n}{mu}\PY{o}{=}\PY{l+m+mf}{1.0}\PY{p}{,} \PY{n}{sigma}\PY{o}{=}\PY{l+m+mf}{0.5}\PY{p}{)} \PY{k}{for} \PY{n}{i} \PY{o+ow}{in} \PY{n+nb}{range}\PY{p}{(}\PY{l+m+mi}{3}\PY{p}{)}\PY{p}{]} \PY{p}{)}
\end{Verbatim}

    \begin{Verbatim}[commandchars=\\\{\}]
[0.6201987142056988, 1.3120253897349055, 0.7844364422139856]
[0.5896986001391438, 1.325007288757538, 1.2354502475094375]
[0.6201987142056988, 1.3120253897349055, 0.7844364422139856]
    \end{Verbatim}

    Note that the sequence of the first three numbers are the same after the
seed is set to the same number. If you choose not to set a seed, the
default behaviour of python is to use the system time to set a seed that
will be different every time you run your code. The purpose of setting
the seed explicitly is to be able to reproduce simulation experiments.

Use \texttt{dir(random)} to see the different methods available to
generate random numbers from different distributions. Use
\texttt{help()} to get helt on any specific method. For instance

    \begin{Verbatim}[commandchars=\\\{\}]
{\color{incolor}In [{\color{incolor}70}]:} \PY{n}{help}\PY{p}{(}\PY{n}{random}\PY{o}{.}\PY{n}{expovariate}\PY{p}{)}
\end{Verbatim}

    \begin{Verbatim}[commandchars=\\\{\}]
Help on method expovariate in module random:

expovariate(lambd) method of random.Random instance
    Exponential distribution.
    
    lambd is 1.0 divided by the desired mean.  It should be
    nonzero.  (The parameter would be called "lambda", but that is
    a reserved word in Python.)  Returned values range from 0 to
    positive infinity if lambd is positive, and from negative
    infinity to 0 if lambd is negative.
    \end{Verbatim}

    \subsection{The SimPy module}\label{the-simpy-module}

There are good introductory
\href{http://simpy.readthedocs.io/en/latest/simpy_intro/index.html}{tutorials},
\href{http://simpy.readthedocs.io/en/latest/examples/index.html}{examples}
and
\href{http://simpy.readthedocs.org/en/latest/api_reference/index.html}{documentation}
for SimPy. In this section we just want to point out a key concept to
understand when implementing simulation models in SimPy.

The \emph{processes} of a discrete event system are implemented as
\emph{generators} in python. The defining property of a generator is
that it can return a value after a call (by use of the \texttt{yield}
keyword), and continue execution where it left off at the next call.
This is exactly what we need to implement processes that performs some
operations and then wait for some event to happen before continuing.

In SimPy, the \texttt{yield} statement in the process generators
\textbf{must} return an event object. The process will continue when the
event happens. Let's look at some simple examples.

    \begin{Verbatim}[commandchars=\\\{\}]
{\color{incolor}In [{\color{incolor}107}]:} \PY{k+kn}{import} \PY{n+nn}{random}
          \PY{k+kn}{import} \PY{n+nn}{simpy}
\end{Verbatim}

    \subsubsection{Example 1 - Reneging
customers}\label{example-1---reneging-customers}

Consider the process of a customer that enters a quee, waits for 1.3
time units and then exits the queue. The process step of waiting is
implemented by returning a \texttt{timeout} event object. The event will
occur when the specified time has passed. We will assme that if the
customer is still in queue when the time has passed, then she will get
impatient and exit the system. Otherwise, we assume that the customer
has already been served or is currently getting served.

    \begin{Verbatim}[commandchars=\\\{\}]
{\color{incolor}In [{\color{incolor}108}]:} \PY{k}{def} \PY{n+nf}{reneging\PYZus{}customer\PYZus{}proc}\PY{p}{(}\PY{n}{env}\PY{p}{,} \PY{n}{name}\PY{p}{,} \PY{n}{patience}\PY{p}{,} \PY{n}{queue}\PY{p}{)}\PY{p}{:}
              \PY{n+nb}{print}\PY{p}{(} \PY{l+s+s2}{\PYZdq{}}\PY{l+s+si}{\PYZpc{}s}\PY{l+s+s2}{ with patience }\PY{l+s+si}{\PYZpc{}f}\PY{l+s+s2}{ enters the queue at time }\PY{l+s+si}{\PYZpc{}f}\PY{l+s+s2}{\PYZdq{}} \PY{o}{\PYZpc{}} \PY{p}{(}\PY{n}{name}\PY{p}{,} \PY{n}{patience}\PY{p}{,} \PY{n}{env}\PY{o}{.}\PY{n}{now}\PY{p}{)} \PY{p}{)}
              \PY{n}{queue}\PY{o}{.}\PY{n}{append}\PY{p}{(}\PY{n}{name}\PY{p}{)} \PY{c+c1}{\PYZsh{} Customers are identified by name, so all names should be unique}
              \PY{k}{yield} \PY{n}{env}\PY{o}{.}\PY{n}{timeout}\PY{p}{(}\PY{l+m+mf}{1.3}\PY{p}{)}
              \PY{k}{if} \PY{n}{name} \PY{o+ow}{in} \PY{n}{queue}\PY{p}{:}
                  \PY{n}{queue}\PY{o}{.}\PY{n}{remove}\PY{p}{(}\PY{n}{name}\PY{p}{)}
                  \PY{n+nb}{print}\PY{p}{(} \PY{l+s+s2}{\PYZdq{}}\PY{l+s+si}{\PYZpc{}s}\PY{l+s+s2}{ got impatient and exited the queue at time }\PY{l+s+si}{\PYZpc{}f}\PY{l+s+s2}{\PYZdq{}} \PY{o}{\PYZpc{}} \PY{p}{(}\PY{n}{name}\PY{p}{,} \PY{n}{env}\PY{o}{.}\PY{n}{now}\PY{p}{)} \PY{p}{)}
              
          \PY{n}{env} \PY{o}{=} \PY{n}{simpy}\PY{o}{.}\PY{n}{Environment}\PY{p}{(}\PY{p}{)}
          \PY{n}{queue} \PY{o}{=} \PY{p}{[}\PY{p}{]}
          
          \PY{n}{env}\PY{o}{.}\PY{n}{process}\PY{p}{(}\PY{n}{reneging\PYZus{}customer\PYZus{}proc}\PY{p}{(}\PY{n}{env}\PY{p}{,} \PY{l+s+s2}{\PYZdq{}}\PY{l+s+s2}{Oscar}\PY{l+s+s2}{\PYZdq{}}\PY{p}{,} \PY{l+m+mf}{1.3}\PY{p}{,} \PY{n}{queue}\PY{p}{)}\PY{p}{)}
          \PY{n}{env}\PY{o}{.}\PY{n}{run}\PY{p}{(}\PY{p}{)}
\end{Verbatim}

    \begin{Verbatim}[commandchars=\\\{\}]
Oscar with patience 1.300000 enters the queue at time 0.000000
Oscar got impatient and exited the queue at time 1.300000
    \end{Verbatim}

    \paragraph{Customer generator}\label{customer-generator}

We will need a process that generates customers that enters the system.
Typically, the time between arrivals is random. Here we assume the time
between arrivals to be exponentially distributed with mean time 1.0.

    \begin{Verbatim}[commandchars=\\\{\}]
{\color{incolor}In [{\color{incolor}109}]:} \PY{k}{def} \PY{n+nf}{customer\PYZus{}generator\PYZus{}proc}\PY{p}{(}\PY{n}{env}\PY{p}{,} \PY{n}{numberOfCustomers}\PY{p}{,} \PY{n}{timeBetween}\PY{p}{,} \PY{n}{queue}\PY{p}{,} \PY{n}{newArrivalEvents}\PY{p}{)}\PY{p}{:}
              \PY{l+s+sd}{\PYZdq{}\PYZdq{}\PYZdq{} Will generate a fixed number of customers, with random time between arrivals.\PYZdq{}\PYZdq{}\PYZdq{}}
              \PY{n}{k} \PY{o}{=} \PY{l+m+mi}{0}
              \PY{k}{while} \PY{n}{k}\PY{o}{\PYZlt{}}\PY{n}{numberOfCustomers}\PY{p}{:}
                  \PY{k}{yield} \PY{n}{env}\PY{o}{.}\PY{n}{timeout}\PY{p}{(} \PY{n}{random}\PY{o}{.}\PY{n}{expovariate}\PY{p}{(}\PY{l+m+mf}{1.0}\PY{o}{/}\PY{n}{timeBetween}\PY{p}{)} \PY{p}{)}
                  \PY{n}{k} \PY{o}{+}\PY{o}{=} \PY{l+m+mi}{1}
                  \PY{n}{env}\PY{o}{.}\PY{n}{process}\PY{p}{(} \PY{n}{reneging\PYZus{}customer\PYZus{}proc}\PY{p}{(}\PY{n}{env}\PY{p}{,} \PY{n}{name} \PY{o}{=} \PY{l+s+s2}{\PYZdq{}}\PY{l+s+s2}{Customer\PYZhy{}}\PY{l+s+si}{\PYZpc{}d}\PY{l+s+s2}{\PYZdq{}} \PY{o}{\PYZpc{}}\PY{k}{k}, patience = 1.3, queue = queue) )
                  \PY{k}{while} \PY{n}{newArrivalEvents} \PY{o}{!=} \PY{p}{[}\PY{p}{]}\PY{p}{:}
                      \PY{n}{ev} \PY{o}{=} \PY{n}{newArrivalEvents}\PY{o}{.}\PY{n}{pop}\PY{p}{(}\PY{l+m+mi}{0}\PY{p}{)}
                      \PY{c+c1}{\PYZsh{} The newArrivalEvents list contains events that servers are waiting for in order to proceed.}
                      \PY{c+c1}{\PYZsh{} What they are waiting for is for a new customer to arrive, so trigger the event}
                      \PY{n}{ev}\PY{o}{.}\PY{n}{succeed}\PY{p}{(}\PY{p}{)}
                      \PY{n+nb}{print}\PY{p}{(} \PY{l+s+s2}{\PYZdq{}}\PY{l+s+s2}{Triggering arrival event}\PY{l+s+s2}{\PYZdq{}}\PY{p}{)}
          
          \PY{n}{env} \PY{o}{=} \PY{n}{simpy}\PY{o}{.}\PY{n}{Environment}\PY{p}{(}\PY{p}{)}
          \PY{n}{queue} \PY{o}{=} \PY{p}{[}\PY{p}{]}
          \PY{n}{waitingForArrivalEvents} \PY{o}{=} \PY{p}{[}\PY{p}{]}
          \PY{n}{env}\PY{o}{.}\PY{n}{process}\PY{p}{(} \PY{n}{customer\PYZus{}generator\PYZus{}proc}\PY{p}{(}\PY{n}{env}\PY{p}{,} \PY{l+m+mi}{8}\PY{p}{,} \PY{l+m+mf}{0.8}\PY{p}{,} \PY{n}{queue}\PY{p}{,} \PY{n}{waitingForArrivalEvents}\PY{p}{)} \PY{p}{)}
          \PY{n}{env}\PY{o}{.}\PY{n}{run}\PY{p}{(}\PY{p}{)}
\end{Verbatim}

    \begin{Verbatim}[commandchars=\\\{\}]
Customer-1 with patience 1.300000 enters the queue at time 1.468228
Customer-2 with patience 1.300000 enters the queue at time 2.243378
Customer-1 got impatient and exited the queue at time 2.768228
Customer-3 with patience 1.300000 enters the queue at time 3.056802
Customer-2 got impatient and exited the queue at time 3.543378
Customer-3 got impatient and exited the queue at time 4.356802
Customer-4 with patience 1.300000 enters the queue at time 5.027282
Customer-4 got impatient and exited the queue at time 6.327282
Customer-5 with patience 1.300000 enters the queue at time 7.311952
Customer-6 with patience 1.300000 enters the queue at time 7.859252
Customer-7 with patience 1.300000 enters the queue at time 8.178111
Customer-8 with patience 1.300000 enters the queue at time 8.350823
Customer-5 got impatient and exited the queue at time 8.611952
Customer-6 got impatient and exited the queue at time 9.159252
Customer-7 got impatient and exited the queue at time 9.478111
Customer-8 got impatient and exited the queue at time 9.650823
    \end{Verbatim}

    \paragraph{Service process}\label{service-process}

There is a service process that takes care of the customers in the
queue. The time to serve a single customer is exponentially distributed
with mean 1.3 time units. The service process takes out customers from
the queueu on a first come first served basis (FIFO queue). If there are
no customers, the server process will wait for the event that a new
customer is arriving.

    \begin{Verbatim}[commandchars=\\\{\}]
{\color{incolor}In [{\color{incolor}110}]:} \PY{k}{def} \PY{n+nf}{service\PYZus{}proc}\PY{p}{(}\PY{n}{env}\PY{p}{,} \PY{n}{serviceTime}\PY{p}{,} \PY{n}{queue}\PY{p}{,} \PY{n}{arrivalEvents}\PY{p}{)}\PY{p}{:}
              \PY{k}{while} \PY{k+kc}{True}\PY{p}{:}
                  \PY{k}{while} \PY{n}{queue} \PY{o}{!=} \PY{p}{[}\PY{p}{]}\PY{p}{:}
                      \PY{n}{customer} \PY{o}{=} \PY{n}{queue}\PY{o}{.}\PY{n}{pop}\PY{p}{(}\PY{l+m+mi}{0}\PY{p}{)} \PY{c+c1}{\PYZsh{} Take out the first customer}
                      \PY{k}{yield} \PY{n}{env}\PY{o}{.}\PY{n}{timeout}\PY{p}{(} \PY{n}{random}\PY{o}{.}\PY{n}{expovariate}\PY{p}{(}\PY{l+m+mf}{1.0}\PY{o}{/}\PY{n}{serviceTime}\PY{p}{)} \PY{p}{)}
                      \PY{n+nb}{print}\PY{p}{(} \PY{l+s+s2}{\PYZdq{}}\PY{l+s+si}{\PYZpc{}s}\PY{l+s+s2}{ served at time }\PY{l+s+si}{\PYZpc{}f}\PY{l+s+s2}{\PYZdq{}} \PY{o}{\PYZpc{}}\PY{p}{(}\PY{n}{customer}\PY{p}{,} \PY{n}{env}\PY{o}{.}\PY{n}{now}\PY{p}{)} \PY{p}{)}
                  \PY{n}{newArrivalEv} \PY{o}{=} \PY{n}{env}\PY{o}{.}\PY{n}{event}\PY{p}{(}\PY{p}{)}
                  \PY{n}{arrivalEvents}\PY{o}{.}\PY{n}{append}\PY{p}{(}\PY{n}{newArrivalEv}\PY{p}{)} 
                  \PY{n+nb}{print}\PY{p}{(} \PY{l+s+s2}{\PYZdq{}}\PY{l+s+s2}{Service process waiting for a new arrival}\PY{l+s+s2}{\PYZdq{}} \PY{p}{)}
                  \PY{k}{yield} \PY{n}{newArrivalEv} \PY{c+c1}{\PYZsh{} Wait for the arrival event to be triggered. This is done in customer\PYZus{}generator\PYZus{}proc}
          
          \PY{n}{env} \PY{o}{=} \PY{n}{simpy}\PY{o}{.}\PY{n}{Environment}\PY{p}{(}\PY{p}{)}
          \PY{n}{queue} \PY{o}{=} \PY{p}{[}\PY{p}{]}
          \PY{n}{waitingForArrivalEvents} \PY{o}{=} \PY{p}{[}\PY{p}{]}
          \PY{n}{env}\PY{o}{.}\PY{n}{process}\PY{p}{(} \PY{n}{service\PYZus{}proc}\PY{p}{(}\PY{n}{env}\PY{p}{,} \PY{l+m+mf}{1.3}\PY{p}{,} \PY{n}{queue}\PY{p}{,} \PY{n}{waitingForArrivalEvents}\PY{p}{)} \PY{p}{)}
          \PY{n}{env}\PY{o}{.}\PY{n}{process}\PY{p}{(} \PY{n}{customer\PYZus{}generator\PYZus{}proc}\PY{p}{(}\PY{n}{env}\PY{p}{,} \PY{l+m+mi}{4}\PY{p}{,} \PY{l+m+mf}{0.8}\PY{p}{,} \PY{n}{queue}\PY{p}{,} \PY{n}{waitingForArrivalEvents}\PY{p}{)} \PY{p}{)}
          
          \PY{n}{env}\PY{o}{.}\PY{n}{run}\PY{p}{(}\PY{p}{)}
\end{Verbatim}

    \begin{Verbatim}[commandchars=\\\{\}]
Service process waiting for a new arrival
Triggering arrival event
Customer-1 with patience 1.300000 enters the queue at time 2.193581
Customer-1 served at time 2.324042
Service process waiting for a new arrival
Triggering arrival event
Customer-2 with patience 1.300000 enters the queue at time 3.629634
Customer-3 with patience 1.300000 enters the queue at time 3.957301
Customer-2 served at time 4.996579
Customer-4 with patience 1.300000 enters the queue at time 5.474184
Customer-4 got impatient and exited the queue at time 6.774184
Customer-3 served at time 7.580712
Service process waiting for a new arrival
    \end{Verbatim}

    \subsubsection{Exercise: What is the probability that a customer will
leave the system without getting
served?}\label{exercise-what-is-the-probability-that-a-customer-will-leave-the-system-without-getting-served}

The code for the simulation model above with reneging customers is
repeated here below, but without the print statements. Think about how
you can record the number of customers that leave without being served
in the simulation. Use this to answer the question.

    \begin{Verbatim}[commandchars=\\\{\}]
{\color{incolor}In [{\color{incolor}104}]:} \PY{k}{def} \PY{n+nf}{reneging\PYZus{}customer\PYZus{}proc}\PY{p}{(}\PY{n}{env}\PY{p}{,} \PY{n}{name}\PY{p}{,} \PY{n}{patience}\PY{p}{,} \PY{n}{queue}\PY{p}{)}\PY{p}{:}
              \PY{n}{queue}\PY{o}{.}\PY{n}{append}\PY{p}{(}\PY{n}{name}\PY{p}{)} \PY{c+c1}{\PYZsh{} Customers are identified by name, so all names should be unique}
              \PY{k}{yield} \PY{n}{env}\PY{o}{.}\PY{n}{timeout}\PY{p}{(}\PY{l+m+mf}{1.3}\PY{p}{)}
              \PY{k}{if} \PY{n}{name} \PY{o+ow}{in} \PY{n}{queue}\PY{p}{:}
                  \PY{n}{queue}\PY{o}{.}\PY{n}{remove}\PY{p}{(}\PY{n}{name}\PY{p}{)}
          
          \PY{k}{def} \PY{n+nf}{customer\PYZus{}generator\PYZus{}proc}\PY{p}{(}\PY{n}{env}\PY{p}{,} \PY{n}{numberOfCustomers}\PY{p}{,} \PY{n}{timeBetween}\PY{p}{,} \PY{n}{queue}\PY{p}{,} \PY{n}{newArrivalEvents}\PY{p}{)}\PY{p}{:}
              \PY{l+s+sd}{\PYZdq{}\PYZdq{}\PYZdq{} Will generate a fixed number of customers, with random time between arrivals.\PYZdq{}\PYZdq{}\PYZdq{}}
              \PY{n}{k} \PY{o}{=} \PY{l+m+mi}{0}
              \PY{k}{while} \PY{n}{k}\PY{o}{\PYZlt{}}\PY{n}{numberOfCustomers}\PY{p}{:}
                  \PY{k}{yield} \PY{n}{env}\PY{o}{.}\PY{n}{timeout}\PY{p}{(} \PY{n}{random}\PY{o}{.}\PY{n}{expovariate}\PY{p}{(}\PY{l+m+mf}{1.0}\PY{o}{/}\PY{n}{timeBetween}\PY{p}{)} \PY{p}{)}
                  \PY{n}{k} \PY{o}{+}\PY{o}{=} \PY{l+m+mi}{1}
                  \PY{n}{env}\PY{o}{.}\PY{n}{process}\PY{p}{(} \PY{n}{reneging\PYZus{}customer\PYZus{}proc}\PY{p}{(}\PY{n}{env}\PY{p}{,} \PY{n}{name} \PY{o}{=} \PY{l+s+s2}{\PYZdq{}}\PY{l+s+s2}{Customer\PYZhy{}}\PY{l+s+si}{\PYZpc{}d}\PY{l+s+s2}{\PYZdq{}} \PY{o}{\PYZpc{}}\PY{k}{k}, patience = 1.3, queue = queue) )
                  \PY{k}{while} \PY{n}{newArrivalEvents} \PY{o}{!=} \PY{p}{[}\PY{p}{]}\PY{p}{:}
                      \PY{n}{ev} \PY{o}{=} \PY{n}{newArrivalEvents}\PY{o}{.}\PY{n}{pop}\PY{p}{(}\PY{l+m+mi}{0}\PY{p}{)}
                      \PY{c+c1}{\PYZsh{} The newArrivalEvents list contains events that servers are waiting for in order to proceed.}
                      \PY{c+c1}{\PYZsh{} What they are waiting for is for a new customer to arrive, so trigger the event}
                      \PY{n}{ev}\PY{o}{.}\PY{n}{succeed}\PY{p}{(}\PY{p}{)}
          
          \PY{k}{def} \PY{n+nf}{service\PYZus{}proc}\PY{p}{(}\PY{n}{env}\PY{p}{,} \PY{n}{serviceTime}\PY{p}{,} \PY{n}{queue}\PY{p}{,} \PY{n}{arrivalEvents}\PY{p}{)}\PY{p}{:}
              \PY{k}{while} \PY{k+kc}{True}\PY{p}{:}
                  \PY{k}{while} \PY{n}{queue} \PY{o}{!=} \PY{p}{[}\PY{p}{]}\PY{p}{:}
                      \PY{n}{customer} \PY{o}{=} \PY{n}{queue}\PY{o}{.}\PY{n}{pop}\PY{p}{(}\PY{l+m+mi}{0}\PY{p}{)} \PY{c+c1}{\PYZsh{} Take out the first customer}
                      \PY{k}{yield} \PY{n}{env}\PY{o}{.}\PY{n}{timeout}\PY{p}{(} \PY{n}{random}\PY{o}{.}\PY{n}{expovariate}\PY{p}{(}\PY{l+m+mf}{1.0}\PY{o}{/}\PY{n}{serviceTime}\PY{p}{)} \PY{p}{)}
                  \PY{n}{newArrivalEv} \PY{o}{=} \PY{n}{env}\PY{o}{.}\PY{n}{event}\PY{p}{(}\PY{p}{)}
                  \PY{n}{arrivalEvents}\PY{o}{.}\PY{n}{append}\PY{p}{(}\PY{n}{newArrivalEv}\PY{p}{)} 
                  \PY{k}{yield} \PY{n}{newArrivalEv} \PY{c+c1}{\PYZsh{} Wait for the arrival event to be triggered. This is done in customer\PYZus{}generator\PYZus{}proc}
          
          \PY{n}{env} \PY{o}{=} \PY{n}{simpy}\PY{o}{.}\PY{n}{Environment}\PY{p}{(}\PY{p}{)}
          \PY{n}{queue} \PY{o}{=} \PY{p}{[}\PY{p}{]}
          \PY{n}{waitingForArrivalEvents} \PY{o}{=} \PY{p}{[}\PY{p}{]}
          \PY{n}{env}\PY{o}{.}\PY{n}{process}\PY{p}{(} \PY{n}{service\PYZus{}proc}\PY{p}{(}\PY{n}{env}\PY{p}{,} \PY{l+m+mf}{1.3}\PY{p}{,} \PY{n}{queue}\PY{p}{,} \PY{n}{waitingForArrivalEvents}\PY{p}{)} \PY{p}{)}
          \PY{n}{env}\PY{o}{.}\PY{n}{process}\PY{p}{(} \PY{n}{customer\PYZus{}generator\PYZus{}proc}\PY{p}{(}\PY{n}{env}\PY{p}{,} \PY{l+m+mi}{4}\PY{p}{,} \PY{l+m+mf}{0.8}\PY{p}{,} \PY{n}{queue}\PY{p}{,} \PY{n}{waitingForArrivalEvents}\PY{p}{)} \PY{p}{)}
          
          \PY{n}{env}\PY{o}{.}\PY{n}{run}\PY{p}{(}\PY{p}{)}
\end{Verbatim}


    % Add a bibliography block to the postdoc
    
    
    
    \end{document}
